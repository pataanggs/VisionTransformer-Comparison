% BAB III - METODOLOGI
\section{Metodologi}

\subsection{Dataset}
% Deskripsi dataset yang digunakan
% Contoh:
% Eksperimen ini menggunakan dataset [nama dataset], yang terdiri dari...
% \begin{itemize}
%     \item Jumlah kelas: 
%     \item Jumlah training samples: 
%     \item Jumlah validation samples: 
%     \item Jumlah test samples: 
%     \item Resolusi gambar: 
% \end{itemize}

\subsection{Preprocessing dan Augmentasi Data}
% Tulis preprocessing dan augmentasi data yang digunakan
% Contoh:
% \subsubsection{Preprocessing}
% \begin{itemize}
%     \item Resize ke resolusi 224x224
%     \item Normalisasi dengan mean=[0.485, 0.456, 0.406] dan std=[0.229, 0.224, 0.225]
% \end{itemize}
% 
% \subsubsection{Augmentasi}
% \begin{itemize}
%     \item Random horizontal flip
%     \item Random rotation (±15 derajat)
%     \item Random crop
%     \item Color jitter
% \end{itemize}

\subsection{Konfigurasi Training}
% Tulis konfigurasi training di sini
% Contoh dalam bentuk tabel:

\begin{table}[h]
\caption{Konfigurasi Training}
\label{tab:config}
\centering
\begin{tabular}{ll}
\hline
\textbf{Parameter} & \textbf{Nilai} \\ \hline
Optimizer & Adam \\
Learning Rate & 0.001 \\
Batch Size & 32 \\
Epochs & 100 \\
Loss Function & Cross Entropy \\
Learning Rate Scheduler & CosineAnnealingLR \\
Weight Decay & 0.0001 \\
\hline
\end{tabular}
\end{table}

\subsection{Library dan Framework}
% Tulis library dan framework yang digunakan
% Contoh:
% \begin{itemize}
%     \item Python 3.8
%     \item PyTorch 1.12
%     \item torchvision 0.13
%     \item timm (PyTorch Image Models)
%     \item NumPy, Pandas, Matplotlib
% \end{itemize}

\subsection{Spesifikasi Hardware}
% Tulis spesifikasi hardware yang digunakan
% Contoh:
% \begin{itemize}
%     \item GPU: NVIDIA RTX 3090 24GB
%     \item CPU: Intel Core i9-10900K
%     \item RAM: 64GB DDR4
%     \item Storage: 1TB NVMe SSD
% \end{itemize}

\subsection{Metrik Evaluasi}
% Tulis cara pengukuran metrik evaluasi
% Contoh:
% Model dievaluasi menggunakan metrik berikut:
% \begin{itemize}
%     \item \textbf{Accuracy}: Persentase prediksi yang benar dari total prediksi
%     \item \textbf{Precision}: True Positive / (True Positive + False Positive)
%     \item \textbf{Recall}: True Positive / (True Positive + False Negative)
%     \item \textbf{F1-Score}: Harmonic mean dari precision dan recall
%     \item \textbf{Inference Time}: Waktu rata-rata untuk inferensi per gambar
%     \item \textbf{Parameters}: Jumlah total parameter model
% \end{itemize}
