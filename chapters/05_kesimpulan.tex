% BAB V - KESIMPULAN DAN SARAN
\section{Kesimpulan dan Saran}

\subsection{Kesimpulan}
% Kesimpulan hasil perbandingan
% Contoh:
% Berdasarkan eksperimen yang telah dilakukan, dapat disimpulkan bahwa:
% \begin{enumerate}
%     \item Model A memberikan akurasi tertinggi dengan [nilai]% namun memiliki parameter terbanyak dan waktu inferensi paling lambat
%     \item Model B menawarkan trade-off terbaik antara akurasi dan efisiensi dengan...
%     \item Model C paling cocok untuk aplikasi real-time karena waktu inferensi tercepat...
% \end{enumerate}

\subsection{Rekomendasi Model}

\subsubsection{Untuk Akurasi Maksimal}
% Rekomendasi model untuk akurasi maksimal
% Contoh:
% Untuk aplikasi yang mengutamakan akurasi tanpa mempertimbangkan batasan komputasi, \textbf{Model X} direkomendasikan karena...

\subsubsection{Untuk Efisiensi Komputasi}
% Rekomendasi model untuk efisiensi komputasi
% Contoh:
% Untuk aplikasi dengan resource terbatas (edge devices, mobile), \textbf{Model Y} direkomendasikan karena...

\subsubsection{Untuk Aplikasi Real-time}
% Rekomendasi model untuk aplikasi real-time
% Contoh:
% Untuk aplikasi yang membutuhkan inferensi cepat seperti video processing real-time, \textbf{Model Z} direkomendasikan karena...

\subsection{Saran untuk Pengembangan Lebih Lanjut}
% Saran untuk penelitian atau pengembangan selanjutnya
% Contoh:
% \begin{enumerate}
%     \item Melakukan eksperimen dengan dataset yang lebih besar dan beragam
%     \item Mengeksplorasi teknik knowledge distillation untuk mengurangi ukuran model tanpa mengorbankan akurasi
%     \item Menguji model dengan quantization untuk meningkatkan kecepatan inferensi
%     \item Melakukan fine-tuning hyperparameter yang lebih mendalam
%     \item Mengimplementasikan teknik ensemble untuk meningkatkan performa
% \end{enumerate}
