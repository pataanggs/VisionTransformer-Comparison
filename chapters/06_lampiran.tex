% BAB VI - LAMPIRAN
\section*{Lampiran}
\addcontentsline{toc}{section}{Lampiran}

\subsection*{A. Source Code}
\addcontentsline{toc}{subsection}{A. Source Code}

% Masukkan source code penting di sini menggunakan lstlisting
% Contoh:

% \subsubsection*{A.1 Konfigurasi Model}
% \begin{lstlisting}[language=Python, caption=Konfigurasi Model ViT]
% import timm
% 
% model = timm.create_model(
%     'vit_base_patch16_224',
%     pretrained=True,
%     num_classes=10
% )
% \end{lstlisting}

% \subsubsection*{A.2 Training Loop}
% \begin{lstlisting}[language=Python, caption=Main Training Loop]
% for epoch in range(num_epochs):
%     model.train()
%     for batch in train_loader:
%         # Training code here
%         pass
% \end{lstlisting}

\subsection*{B. Output Training Log}
\addcontentsline{toc}{subsection}{B. Output Training Log}

% Masukkan output training log atau tabel training progress
% Contoh:

% \begin{table}[h]
% \caption{Training Log Model Terbaik}
% \label{tab:training-log}
% \centering
% \begin{tabular}{cccccc}
% \hline
% \textbf{Epoch} & \textbf{Train Loss} & \textbf{Train Acc} & \textbf{Val Loss} & \textbf{Val Acc} & \textbf{LR} \\ \hline
% 1 & 2.301 & 10.5 & 2.298 & 11.2 & 0.001 \\
% 10 & 1.526 & 45.8 & 1.612 & 42.3 & 0.001 \\
% 50 & 0.342 & 88.6 & 0.456 & 85.4 & 0.0001 \\
% 100 & 0.123 & 96.2 & 0.234 & 92.8 & 0.00001 \\
% \hline
% \end{tabular}
% \end{table}

\subsection*{C. Screenshot Hasil Eksperimen}
\addcontentsline{toc}{subsection}{C. Screenshot Hasil Eksperimen}

% Masukkan screenshot hasil eksperimen
% Contoh:

% \begin{figure}[h]
%     \centering
%     \includegraphics[width=0.9\textwidth]{Figure/experiment_dashboard.png}
%     \caption{Dashboard Eksperimen (Weights \& Biases)}
%     \label{fig:dashboard}
% \end{figure}

% \begin{figure}[h]
%     \centering
%     \includegraphics[width=0.9\textwidth]{Figure/tensorboard_screenshot.png}
%     \caption{Visualisasi Training di TensorBoard}
%     \label{fig:tensorboard}
% \end{figure}
