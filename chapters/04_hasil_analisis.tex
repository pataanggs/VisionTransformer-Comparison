% BAB IV - HASIL DAN ANALISIS
\section{Hasil dan Analisis}

\subsection{Perbandingan Jumlah Parameter}
% Tabel perbandingan jumlah parameter

\begin{table}[h]
\caption{Perbandingan Jumlah Parameter}
\label{tab:params}
\centering
\begin{tabular}{lrr}
\hline
\textbf{Model} & \textbf{Parameters (M)} & \textbf{FLOPs (G)} \\ \hline
Model 1 & - & - \\
Model 2 & - & - \\
Model 3 & - & - \\
\hline
\end{tabular}
\end{table}

\subsection{Perbandingan Metrik Performa}
% Tabel perbandingan metrik performa

\begin{table}[h]
\caption{Perbandingan Metrik Performa}
\label{tab:performance}
\centering
\begin{tabular}{lcccc}
\hline
\textbf{Model} & \textbf{Accuracy (\%)} & \textbf{Precision (\%)} & \textbf{Recall (\%)} & \textbf{F1-Score (\%)} \\ \hline
Model 1 & - & - & - & - \\
Model 2 & - & - & - & - \\
Model 3 & - & - & - & - \\
\hline
\end{tabular}
\end{table}

\subsection{Perbandingan Waktu Inferensi}
% Tabel perbandingan waktu inferensi

\begin{table}[h]
\caption{Perbandingan Waktu Inferensi}
\label{tab:inference}
\centering
\begin{tabular}{lcc}
\hline
\textbf{Model} & \textbf{Inference Time (ms)} & \textbf{Throughput (images/s)} \\ \hline
Model 1 & - & - \\
Model 2 & - & - \\
Model 3 & - & - \\
\hline
\end{tabular}
\end{table}

\subsection{Visualisasi}

\subsubsection{Kurva Learning}
% Gambar kurva learning (training vs validation loss/accuracy)
% Contoh:
% \begin{figure}[h]
%     \centering
%     \includegraphics[width=0.8\textwidth]{Figure/learning_curves.png}
%     \caption{Kurva Learning: Training dan Validation Loss}
%     \label{fig:learning-curves}
% \end{figure}

\subsubsection{Confusion Matrix}
% Gambar confusion matrix
% Contoh:
% \begin{figure}[h]
%     \centering
%     \includegraphics[width=0.6\textwidth]{Figure/confusion_matrix.png}
%     \caption{Confusion Matrix Model Terbaik}
%     \label{fig:confusion-matrix}
% \end{figure}

\subsubsection{Contoh Prediksi}
% Gambar contoh prediksi (benar dan salah)
% \begin{figure}[h]
% 	\centering
% 	\begin{subfigure}[b]{0.3\textwidth}
% 		\centering
% 		\includegraphics[width=1\textwidth]{Figure/prediction1.png}
% 		\caption{Prediksi Benar}
% 		\label{fig:pred-correct}
% 	\end{subfigure}
% 	\hfill
% 	\begin{subfigure}[b]{0.3\textwidth}
% 		\centering
% 		\includegraphics[width=1\textwidth]{Figure/prediction2.png}
% 		\caption{Prediksi Salah}
% 		\label{fig:pred-wrong}
% 	\end{subfigure}
% 	\caption{Contoh Hasil Prediksi}
% 	\label{fig:predictions}
% \end{figure}

\subsection{Analisis Mendalam}

\subsubsection{Analisis Performa Model}
% Analisis mengapa model A lebih baik dari model B
% Contoh:
% Model A menunjukkan performa yang lebih baik dibandingkan Model B dalam hal akurasi karena...

\subsubsection{Trade-off Akurasi, Parameter, dan Kecepatan}
% Analisis trade-off
% Contoh:
% \begin{itemize}
%     \item Model dengan parameter terbanyak (Model X) memberikan akurasi tertinggi namun...
%     \item Model dengan parameter paling sedikit (Model Y) memiliki waktu inferensi tercepat...
% \end{itemize}

\subsubsection{Kesesuaian Model dengan Dataset}
% Analisis kesesuaian model dengan dataset yang digunakan
% Contoh:
% Berdasarkan karakteristik dataset yang digunakan, Model Z paling sesuai karena...
